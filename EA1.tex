\documentclass[12pt]{article}
\usepackage[utf8]{inputenc}
\usepackage[T1]{fontenc}
\usepackage{lmodern}
\usepackage{ngerman}
\usepackage{amsmath}
\usepackage{amssymb}
\usepackage{array}
\usepackage{german, fancyheadings}
\usepackage{eurosym}
\pagestyle{fancy}
\setlength{\parindent}{0em}
\newcommand{\N}{\mathbb{N}}
\newcommand{\K}{\mathbb{K}}
\newcommand{\Z}{\mathbb{Z}}
\newcommand{\M}{$\times$}
\newtheorem{Sa}{Satz}[subsection]
\newtheorem{Kor}{Korollar}[subsection]
\newtheorem{Prop}{Proposition}[subsection]
\newtheorem{Def}{Definition}[subsection]
\lhead{Marcel Benders - Matrikelnummer: 5431760}
\begin{document}
\section*{1.2}
Den Beweis erbringen wir mithilfe von Wahrheitstafeln. \\

1: 


\begin{tabular}{|c|c|c|c||c|}
\hline
$A$ & $B$   & $C$   & $B \vee C$    & $A \wedge (B \vee C)$  \\
\hline
\hline
0   &  0    & 0     &   0           &   0\\
0   &  0    & 1     &   1           &   0\\
0   &  1    & 0     &   1           &   0\\
0   &  1    & 1     &   1           &   0\\
1   &  0    & 0     &   0           &   0\\
1   &  0    & 1     &   1           &   1\\
1   &  1    & 0     &   1           &   1\\
1   &  1    & 1     &   1           &   1\\
\hline

\end{tabular}




\begin{tabular}{|c|c|c|c|c||c|}
\hline
$A$ & $B$   & $C$   & $A \wedge B$  & $A \wedge C$  & $(A \wedge B) \vee (A \wedge C)$  \\
\hline
\hline
0   &  0    & 0     &   0           &   0           & 0 \\
0   &  0    & 1     &   0           &   0           & 0 \\
0   &  1    & 0     &   0           &   0           & 0 \\
0   &  1    & 1     &   0           &   0           & 0 \\
1   &  0    & 0     &   0           &   0           & 0 \\
1   &  0    & 1     &   0           &   1           & 1 \\
1   &  1    & 0     &   1           &   0           & 1 \\
1   &  1    & 1     &   1           &   1           & 1 \\
\hline

\end{tabular}

Die beiden hinteren Spalten sind identisch, somit sind die beiden Formeln logisch äquivalent.

\newpage

2:

\begin{tabular}{|c|c|c|c||c|}
\hline
$A$ & $B$   & $C$   & $B \wedge C$    & $A \vee (B \wedge C)$  \\
\hline
\hline
0   &  0    & 0     &   0           &   0\\
0   &  0    & 1     &   0           &   0\\
0   &  1    & 0     &   0           &   0\\
0   &  1    & 1     &   1           &   1\\
1   &  0    & 0     &   0           &   1\\
1   &  0    & 1     &   0           &   1\\
1   &  1    & 0     &   0           &   1\\
1   &  1    & 1     &   1           &   1\\
\hline

\end{tabular}




\begin{tabular}{|c|c|c|c|c||c|}
\hline
$A$ & $B$   & $C$   & $A \vee B$  & $A \vee C$  & $(A \vee B) \wedge (A \vee C)$  \\
\hline
\hline
0   &  0    & 0     &   0           &   0           & 0 \\
0   &  0    & 1     &   0           &   1           & 0 \\
0   &  1    & 0     &   1           &   0           & 0 \\
0   &  1    & 1     &   1           &   1           & 1 \\
1   &  0    & 0     &   1           &   1           & 1 \\
1   &  0    & 1     &   1           &   1           & 1 \\
1   &  1    & 0     &   1           &   1           & 1 \\
1   &  1    & 1     &   1           &   1           & 1 \\
\hline

\end{tabular}

Auch hier vergleichen wir wieder die beiden Ergebnisspalten und können ablesen, dass die
Formeln logisch äquivalent sind.

\section*{1.3}
\subsection*{1}
Zu Zeigen: $f$ ist surjektiv.

Gegeben ein beliebiges $a \in \K$ und setzen $d = 1, c, d = 0$ in Matrix $A$ ein. Es folgt mit

\begin{eqnarray*}
A = \left( 
    \begin{array}{ll} 
    a & 0 \\
    0 & 1
    \end{array} \right) \in M_{22}(\K) \\
f(
\left(
\begin{array}{ll}
a & 0 \\
0 & 1
\end{array}
\right)
) = a \cdot 1 - 0 \cdot 0,
\end{eqnarray*}

dass $f$ surjektiv ist.

\subsection*{2}
Zu Zeigen: $f \not=$ injektiv.

Gegeben sind folgende Matritzen;

\begin{equation*}
A = \left( 
    \begin{array}{ll} 
    2 & 2 \\
    1 & 2
    \end{array} \right), B = \left( 
    \begin{array}{ll} 
    2 & 0 \\
    0 & 1
    \end{array} \right) \in M_{22}(\K).
\end{equation*}

Diese setzen wir in die Funktion ein und erhalten jeweils

\begin{equation*}
f^1( \left( 
    \begin{array}{ll} 
    2 & 2 \\
    1 & 2
    \end{array} \right) ) = 2 \cdot 2 - 1 \cdot 2 = 2,  
\end{equation*}


\begin{equation*}
f^2( \left ( 
    \begin{array}{ll} 
    2 & 0 \\
    0 & 1
    \end{array} \right) ) = 2 \cdot 1 - 0 \cdot 0 = 2.
\end{equation*}

Es gilt somit $f^1 = f^2 \Rightarrow f \not=$ injektiv.

\subsection*{3}
Seien $A = \left( \begin{array}{ll} a & b \\ c & d   \end{array} \right), 
B = \left( \begin{array}{ll} a' & b' \\ c' & d'   \end{array} \right) \in M_{22}(\K)$

Dann gilt:

\begin{eqnarray*}
f(AB) &=& f( \left(  
    \begin{array}{ll} 
    a & b \\
    c & d   
    \end{array}\right)
    \cdot
    \left( 
    \begin{array}{ll} 
    a' & b' \\
    c' & d'  
    \end{array} \right)     
     )\\
    &=& f(
    \left( 
        \begin{array}{ll} 
        aa'+bc' & ab'+bd' \\
         a'c+c'd & cb'+dd'  
         \end{array} \right)  
     ) \\
     &=& (aa'+bc')(cb'+dd')-(ab'+bd')(a'c+c'd)\\
     &=&
     (ad - bc) \cdot (a'd' - b'c') \\
     &=&  f(
         \left(  
    \begin{array}{ll} a & b \\
     c & d   \end{array}\right) 
     )  f(
             \left( 
        \begin{array}{ll} 
        a' & b' \\
         c' & d'  
         \end{array} \right)     
     ) \\
     &=& f(A)f(B).
\end{eqnarray*}

Somit gilt die Behauptung $f(AB) = f(A)f(B).$

\subsection*{4}
Seien $A = \left( \begin{array}{ll} a & b \\ c & d   \end{array} \right), 
B = \left( \begin{array}{ll} a' & b' \\ c' & d'   \end{array} \right) \in M_{22}(\K)$

Dann gilt:

\begin{eqnarray*}
f(A+B) &=& f( \left(  
    \begin{array}{ll} a & b \\
     c & d   \end{array}\right)
     +
    \left( 
        \begin{array}{ll} 
        a' & b' \\
         c' & d'  
         \end{array} \right)     
     )\\
     &=& f(
    \left( 
        \begin{array}{ll} 
        a+a' & b+b' \\
         c+c' & d+d'  
         \end{array} \right)  
     ) \\
     &=& (a+a')(d+d')-(b+b')(c+c')\\
     &=&
     ad - bc + a'd' - b'c' \\
     &=&  f(
         \left(  
    \begin{array}{ll} a & b \\
     c & d   \end{array}\right) 
     ) + f(
             \left( 
        \begin{array}{ll} 
        a' & b' \\
         c' & d'  
         \end{array} \right)     
     ) \\
     &=& f(A)+f(B).
\end{eqnarray*}

Somit gilt die Behauptung $f(A+B) = f(A)+f(B).$

\section*{1.4}
--

\section*{1.5}
--
\newpage
\section*{1.6}
\subsection*{1}
Zu Zeigen: binäre Folgen der Länge 1, 2, 3:
\begin{eqnarray*}
\lambda_1 &=& \{(0), (1)\} \\
\lambda_2 &=& \{(0,0), (0,1), (1,0), (1,1)\} \\
\lambda_3 &=& \{(0,0,0),(0,0,1),(0,1,0),(0,1,1),(1,0,0),(1,0,1),(1,1,0),(1,1,1)\}
\end{eqnarray*}

\subsection*{2}
Mit der Funktion 
$$f:\N \rightarrow \N \text{ mit } f(n) = 2^n \text{ für alle }n \in \N$$

erhält man die Anzahl binärer Folgen der Länge $n$:

\begin{equation*}
\{(x_1, \dots , x_n)_1, \dots ,(x_1, \dots, x_n)_{2^n} \} 
\end{equation*}


 Dies werden wir mit der vollständigen Induktion
nach $n$ beweisen. \\

\textbf{Induktionsanfang: } Sei $n_0=1$, dann gilt mit
\begin{equation*}
2^1 = 2 = |\lambda_1|, \lambda_1 = \{(0), (1)\}
\end{equation*}
 der Induktionsanfang. \\

 \textbf{Induktionsannahme: } Für alle $n \ge n_0$ gilt 
 \begin{equation*}
 2^n = \{(x_1, \dots , x_n)_1, \dots ,(x_1, \dots, x_n)_{2^n} \} ,
\end{equation*}
was im Induktionsschritt zu beweisen ist. \\

\textbf{Induktionsschritt: } Zu Zeigen ist, dass

\begin{equation*}
 2^{n+1} = \{(x_1, \dots , x_{n+1})_1, \dots ,(x_1, \dots, x_{n+1})_{2^{n+1}} \}
\end{equation*}

gilt was folgt:

 \begin{eqnarray*}
  \{(x_1, \dots , x_{n+1})_1, \dots ,(x_1, \dots, x_{n+1})_{2^{n+1}} \} &=& 2^{n+1}\\
 &=& 2^n \cdot 2  \\
  &\underbrace{=}_{I.A.}&  \underbrace{\{(x_1, \dots , x_n)_1, \dots ,(x_1, \dots, x_n)_{2^n} \}}_{=2^n}  \cdot 2
\end{eqnarray*}

Hierraus ist ersichtlich, das die Menge der Folgen mit jedem Schritt $n+1$ um den Faktor 2 wächst.

Mit dem Satz der vollständigen Induktion folgt die Behauptung.
\end{document}
