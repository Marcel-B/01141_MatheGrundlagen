\documentclass[12pt]{article}
\usepackage[utf8]{inputenc}
\usepackage[T1]{fontenc}
\usepackage{lmodern}
\usepackage{ngerman}
\usepackage{amsmath}
\usepackage{amssymb}
\usepackage{array}
\usepackage{german, fancyheadings}
\usepackage{eurosym}
\pagestyle{fancy}
\setlength{\parindent}{0em}
\newcommand{\N}{\mathbb{N}}
\newcommand{\M}{$\times$}
\newtheorem{Sa}{Satz}[subsection]
\newtheorem{Kor}{Korollar}[subsection]
\newtheorem{Prop}{Proposition}[subsection]
\newtheorem{Def}{Definition}[subsection]
\lhead{Marcel Benders - Matrikelnummer: 5431760}
\begin{document}
\section*{1.2}
Den Beweis erbringen wir mithilfe von Wahrheitstafeln. \\

1: 


\begin{tabular}{|c|c|c|c||c|}
\hline
$A$ & $B$   & $C$   & $B \vee C$    & $A \wedge (B \vee C)$  \\
\hline
\hline
0   &  0    & 0     &   0           &   0\\
0   &  0    & 1     &   1           &   0\\
0   &  1    & 0     &   1           &   0\\
0   &  1    & 1     &   1           &   0\\
1   &  0    & 0     &   0           &   0\\
1   &  0    & 1     &   1           &   1\\
1   &  1    & 0     &   1           &   1\\
1   &  1    & 1     &   1           &   1\\
\hline

\end{tabular}




\begin{tabular}{|c|c|c|c|c||c|}
\hline
$A$ & $B$   & $C$   & $A \wedge B$  & $A \wedge C$  & $(A \wedge B) \vee (A \wedge C)$  \\
\hline
\hline
0   &  0    & 0     &   0           &   0           & 0 \\
0   &  0    & 1     &   0           &   0           & 0 \\
0   &  1    & 0     &   0           &   0           & 0 \\
0   &  1    & 1     &   0           &   0           & 0 \\
1   &  0    & 0     &   0           &   0           & 0 \\
1   &  0    & 1     &   0           &   1           & 1 \\
1   &  1    & 0     &   1           &   0           & 1 \\
1   &  1    & 1     &   1           &   1           & 1 \\
\hline

\end{tabular}

Die beiden hinteren Spalten sind identisch, somit sind die beiden Formeln logisch äquivalent.

\newpage

2:

\begin{tabular}{|c|c|c|c||c|}
\hline
$A$ & $B$   & $C$   & $B \wedge C$    & $A \vee (B \wedge C)$  \\
\hline
\hline
0   &  0    & 0     &   0           &   0\\
0   &  0    & 1     &   0           &   0\\
0   &  1    & 0     &   0           &   0\\
0   &  1    & 1     &   1           &   1\\
1   &  0    & 0     &   0           &   1\\
1   &  0    & 1     &   0           &   1\\
1   &  1    & 0     &   0           &   1\\
1   &  1    & 1     &   1           &   1\\
\hline

\end{tabular}




\begin{tabular}{|c|c|c|c|c||c|}
\hline
$A$ & $B$   & $C$   & $A \vee B$  & $A \vee C$  & $(A \vee B) \wedge (A \vee C)$  \\
\hline
\hline
0   &  0    & 0     &   0           &   0           & 0 \\
0   &  0    & 1     &   0           &   1           & 0 \\
0   &  1    & 0     &   1           &   0           & 0 \\
0   &  1    & 1     &   1           &   1           & 1 \\
1   &  0    & 0     &   1           &   1           & 1 \\
1   &  0    & 1     &   1           &   1           & 1 \\
1   &  1    & 0     &   1           &   1           & 1 \\
1   &  1    & 1     &   1           &   1           & 1 \\
\hline

\end{tabular}

Auch hier vergleichen wir wieder die beiden Ergebnisspalten und können ablesen, dass die
Formeln logisch äquivalent sind.

\section*{1.3}
--

\section*{1.4}
--

\section*{1.5}
--

\section*{1.6}
--

\end{document}
